\documentclass[11pt]{article}

% 必备宏包
\usepackage{amsmath}
\usepackage{listings}
\usepackage{xcolor}
\usepackage{booktabs}
\usepackage{setspace}
\usepackage[hidelinks]{hyperref}
\usepackage{graphicx}
\usepackage{float}
\usepackage[most]{tcolorbox}
\usepackage{ulem} % For strikethrough text (\sout)

% 设置图片路径(当前目录)
\graphicspath{{./}}

% 中文支持
\usepackage{ctex} % 确保系统安装了中文字体
\setCJKmainfont[BoldFont={SimHei}]{SimSun} % 设置中文主字体(宋体),使用黑体作为粗体
\setCJKsansfont{SimHei} % 设置中文无衬线字体(黑体)
\setCJKmonofont{FangSong} % 设置中文等宽字体(仿宋)
\XeTeXinputencoding "utf8" % UTF-8 编码

% 页面布局
\usepackage[a4paper, left=2.5cm, right=2.5cm, top=3cm, bottom=2.5cm]{geometry}
\usepackage{fancyhdr}
\pagestyle{fancy}
\lhead{\textbf{Brighton}}
\chead{Image Library}
\rhead{CS219 Project4}

% 代码块样式
\definecolor{bgcolor}{RGB}{230,230,230}
\definecolor{bgcolor2}{RGB}{244,230,244}
\definecolor{licolor}{RGB}{100,100,100}
\newtcblisting{commandline}{
  breakable,
  colback=bgcolor,
  colframe=licolor,
  listing only,
  arc=5mm,
  boxrule=1pt,
  left=6pt, right=6pt, top=6pt, bottom=6pt,
}
\newtcblisting{codeline}{
  breakable,
  colback=bgcolor2,
  colframe=licolor,
  listing only,
  arc=7mm,
  boxrule=1pt,
  left=5pt, right=5pt, top=5pt, bottom=5pt,
  listing options={
    basicstyle=\ttfamily\fontsize{10pt}{11pt}\selectfont,
    language=bash,
    showstringspaces=false,
    aboveskip=1mm, belowskip=1mm,
    columns=flexible,
    keepspaces=true,
    escapeinside={(*@}{@*)},
    numbers=none,
    frame=none,
  },
  before=\setstretch{1.1},
}

% 标题信息
\title{\textbf{A Simple Image Library} }
\author{}
\date{}

% 自定义标题布局
\makeatletter
\renewcommand{\maketitle}{
  \begin{center}
    {\LARGE \@title \par}
    \vspace{1em}
    \textbf{\@author}
    \vspace{1em}
    \begin{tabular}{ll}
      Student ID: & 12312710 \\
      Name:       & 胥熠/Brighton \\
      Course:     & CS219: Advanced Programming \\
      Date:       & \today
    \end{tabular}
    \vspace{0.5em}
    \hrule height 0.8pt
  \end{center}
}
\makeatother

% 文档主体
\begin{document}
\thispagestyle{empty}
\maketitle

{
\setstretch{1.2}
\tableofcontents
}
\vspace{3em}



\newpage % 强制换页
\section{前言}
Project4作为一个保留节目,延续了前几个学期实现一个C++矩阵类的传统要求,\\
\sout{似乎是因为于老师似乎比较喜欢cv::Mat}。
这个Project看似简单,实则暗藏玄机:内存是否安全申请和释放?类的设计是否高效优雅,还能满足鲁棒性,易用性?想要同时达成这些条件,其实相对来说是比较挑战的。
以及如何在实现一个类的同时,引入一些有意思的东西呢?

\section{功能展示}
\subsection{文件结构}
本图像库主要由以下核心文件构成:
\begin{itemize}
  \item \texttt{image.h}:图像类的头文件,包含了图像类的声明和相关函数的声明,模版类函数的实现。
  \item \texttt{image\_proc.cpp}:图像处理函数的实现文件,包含了部分对图像进行处理的函数。
  \item \texttt{image\_io.cpp}:图像输入输出函数的实现文件,包含了读取和保存图像的函数。
\end{itemize}
辅助文件包括:
\begin{itemize}
    \item \texttt{demo.cpp}:一个演示程序,展示了库的主要功能和用法。
    \item \texttt{CMakeLists.txt}:用于构建库和演示程序的CMake配置文件。
\end{itemize}

\subsection{运行环境}
本库使用C++17标准编写。运行设备为Lenovo Legion Y9000P运行x86下的Windows 11, 使用Intel i9-13900HX(8 P-cores + 16 E-cores),24GB DDR5运存,支持OpenMP以及AVX2指令集。
编译步骤为(上述文件均在path路径下,命令行为Ubuntu 20.04的wsl):
\begin{commandline}
wsl *path*$ mkdir build
wsl *path*$ cd build
wsl *path*/build$ cmake .. -DCMAKE_BUILD_TYPE=Release
wsl *path*/build$ cmake --build .
wsl *path*/build$ ./demo 
wsl *path*/build$ ./demo input.bmp input2.bmp 
\end{commandline}
如果运行设备不支持OpenMP和AVX2指令集,程序不会编译错误,而会在编译时候提供一个warning: \#warning "AVX2 not enabled" [-Wcpp]。同时在运行时不会运行相关代码。

\subsection{库调用指南}
首先,我们可以导入一张 24位无压缩BMP 图像,获取其基本属性,并在处理后保存。

\subsubsection{核心对象解释: Pixel 与 Image}
库的核心是两个模板类:
\begin{itemize}
  \item \texttt{ILib::Pixel<T, Channels>}:表示一个像素,其中 \texttt{T} 是每个通道的数据类型(如 \texttt{unsigned char}, \texttt{float}),\texttt{Channels} 是通道数。
  \item \texttt{ILib::Image<PixelType>}:表示一个图像,其中 \texttt{PixelType} 是具体的像素类型(例如 \texttt{ILib::Pixel<unsigned char, 3>} )。
\end{itemize}
同时,库的namespace为ILib。如果我们想要创造一个24位的RGB图像存储对象,我们可以使用ILib::Image<ILib::Pixel<unsigned char,3>>。
也可以使用ILib::ImageU8C3(内置using定义)来表示。

\subsubsection{图像加载查询与保存}
首先,如需要使用图像库,我们需要包含库头文件:\#include "image.h"。
\begin{figure}[H]
  \centering
  \includegraphics[width=0.4\textwidth]{input.bmp}
  \caption{input.bmp(后续处理使用的图像)}
  \label{fig:input}
\end{figure}

\begin{commandline}
    ILib::ImageU8C3 image = ILib::loadImageFromFile("input.bmp");
    std::cout << "Image loaded: " << input_image.width() << "x" << input_image.height()
    << " Channels: " << input_image.getPixelTypeChannels()
    << " ColorSpace: " << static_cast<int>(input_image.getColorSpace()) << std::endl;
    // 一些处理 
    ILib::saveImageToFile(image, "output.bmp");
    // 其实库不需要用户手动销毁对象,对象会在生命周期结束后自动调用release方法。
    image.release();
\end{commandline}


\subsubsection{图像操作}

\textbf{亮度调整}:支持通过adjustBrightness、adjustBrightness\_simd函数来实现,也可以通过更简单的 += 操作符来实现。

\begin{commandline}
    image += 50;
(or)image.adjustBrightness(50); 
(or)image.adjustBrightness_simd(50);
\end{commandline}


\begin{figure}[H]
  \centering
  % 第一个图片所在的区域,宽度设置为文本宽度的45%
  \begin{minipage}[b]{0.49\textwidth}
    \centering
    \includegraphics[width=0.4\textwidth]{bright.png}
   \caption{亮度调整}
    \label{fig:example}
  \end{minipage}
  \hfill % 用于在两图片之间插入间距
  % 第二个图片所在的区域,宽度同样为45%
  \begin{minipage}[b]{0.49\textwidth}
    \centering
    \includegraphics[width=0.4\textwidth]{blend.png}
  \caption{图像融合}
    \label{fig:example}
  \end{minipage}
\end{figure}

\textbf{图像融合}:支持对两张基本参数相同的图像进行混合,库提供的 \texttt{blend} 函数允许通过 alpha 权重控制两图的混合比例。操作符 \texttt{+} 也被重载用于两图的等权重混合。
\begin{commandline}
    ILib::ImageU8C3 image2 = ILib::loadImageFromFile("input2.bmp");
    // 权重为:70% image + 30% image2
    ILib::ImageU8C3 blended_image = ILib::blend(image, image2, 0.7f);
    ILib::ImageU8C3 blended_image2 = image + image2; //权重各50%
\end{commandline}



\textbf{灰度转换}:支持对24位BMP图像进行灰度化处理,处理后的图像类型为8位灰度图像(通道数为1)。同时也支持保存为灰度图像。
\begin{commandline}
    ILib::ImageU8C1 image_gray = image.toGray(); // 转换为灰度图像
    ILib::saveImageToFile(image_gray, "output_gray.bmp"); // 支持保存为灰度图像
\end{commandline}

\begin{figure}[H]
  \centering
  % 第一个图片所在的区域,宽度设置为文本宽度的45%
  \begin{minipage}[b]{0.49\textwidth}
    \centering
    \includegraphics[width=0.4\textwidth]{gray.png}
   \caption{灰度图}
    \label{fig:example}
  \end{minipage}
  \hfill % 用于在两图片之间插入间距
  % 第二个图片所在的区域,宽度同样为45%
  \begin{minipage}[b]{0.49\textwidth}
    \centering
    \includegraphics[width=0.4\textwidth]{roi.png}
  \caption{ROI操作}
    \label{fig:example}
  \end{minipage}
\end{figure}

\textbf{区域图像(Region of Interest)操作}:可以高效的选中图像的特点子区域,而无需复制像素数据。同时支持基于此区域对原图进行修改。
\begin{commandline}
    //定义一个ROI区域,这里是图像的中心四分之一部分。
    Rect roi = {image.width()/4, image.height()/4, 
                image.width()/2, image.height()/4};
    // 获取ROI区域
    ILib::ImageU8C3 roi_view = image_for_roi.roi(rect); 
    roi_view += -50;
    ILib::saveImageToFile(image, "output_roi.bmp"); 
\end{commandline}


\subsubsection{卷积与滤波}
支持通用的2D卷积操作 (\texttt{filter2D}),并基于此提供了一系列预定义的图像滤波效果,
如高斯模糊、锐化、Sobel边缘检测和浮雕效果,部分预定核的参数可修改。用户也可以自定义卷积核参与计算。
\begin{commandline}
  // 5x5 高斯核,标准差 1.5,其中5,1.5可修改
  ILib::ImageU8C3 blurred_image = ILib::gaussianBlur(image, 5, 1.5);
  // 3x3 Sobel核,阈值 120,true为采用曼哈顿距离,false为欧几里得距离
  ILib::ImageU8C1 image_gray = ILib::convertToGrayscale(image);
  ILib::ImageU8C1 image_sobel_edge;
  ILib::sobelMagnitude(image_gray, image_sobel_edge, 3, 120.0, true);
  // 锐化
  ILib::ImageU8C3 sharpened_image = ILib::sharpen(image);
  // 浮雕
  ILib::ImageU8C3 embossed_image = ILib::emboss(image);
  // 自定义卷积核示例
  ILib::Image<ILib::Pixel<float, 1>> my_krnl(3, 3, ILib::ColorSpace::UNKNOWN);
  my_krnl.at(0,0) = {1.1f}; my_krnl(0,1) = {4.5f}; my_krnl(0,2) = {1.4f};
  my_krnl.at(1,0) = {1.0f}; my_krnl(1,1) = {-9.0f};my_krnl(1,2) = {1.0f}; 
  my_krnl.at(2,0) = {0.9f}; my_krnl(2,1) = {0.8f}; my_krnl(2,2) = {1.0f};
  ILib::ImageU8C3 custom_result;
  ILib::filter2D(input_image, custom_result, custom_kernel);
\end{commandline}

\begin{figure}[H]
  \centering
  % 第一个图片所在的区域,宽度设置为文本宽度的45%
  \begin{minipage}[b]{0.49\textwidth}
    \centering
    \includegraphics[width=0.4\textwidth]{gaussian.png}
   \caption{高斯模糊}
    \label{fig:example}
  \end{minipage}
  \hfill % 用于在两图片之间插入间距
  % 第二个图片所在的区域,宽度同样为45%
  \begin{minipage}[b]{0.49\textwidth}
    \centering
    \includegraphics[width=0.4\textwidth]{sobel.png}
  \caption{Sobel边缘检测}
    \label{fig:example}
  \end{minipage}
\end{figure}

\begin{figure}[H]
  \centering
  % 第一个图片所在的区域,宽度设置为文本宽度的45%
  \begin{minipage}[b]{0.49\textwidth}
    \centering
    \includegraphics[width=0.4\textwidth]{sharpen.png}
   \caption{锐化}
    \label{fig:example}
  \end{minipage}
  \hfill % 用于在两图片之间插入间距
  % 第二个图片所在的区域,宽度同样为45%
  \begin{minipage}[b]{0.49\textwidth}
    \centering
    \includegraphics[width=0.4\textwidth]{emboss.png}
  \caption{浮雕}
    \label{fig:example}
  \end{minipage}
\end{figure}


\begin{figure}[H]
  \centering
  % 第一个图片所在的区域,宽度设置为文本宽度的45%
  \begin{minipage}[b]{0.49\textwidth}
    \centering
    \includegraphics[width=0.4\textwidth]{custom1.png}
   \caption{自定义卷积核(已黑化)}
    \label{fig:example}
  \end{minipage}
  \hfill % 用于在两图片之间插入间距
  % 第二个图片所在的区域,宽度同样为45%
  \begin{minipage}[b]{0.49\textwidth}
    \centering
    \includegraphics[width=0.4\textwidth]{custom2.png}
  \caption{自定义卷积核(已飞升)}
    \label{fig:example}
  \end{minipage}
\end{figure}

\subsubsection{错误处理}
库设计了很多编译阶段的检查,例如对于Pixel对象来说,会检查模板传入参数合理性:
\begin{codeline}
  static_assert(std::is_arithmetic_v<T> || std::is_enum_v<T>, 
                "Pixel type must be arithmetic or enum");
  static_assert(Channels > 0, "Pixel must have at least one channel");
\end{codeline}
如果传入的参数不合理,则会在编译阶段会报错。


在include SIMD或者AVX2指令集函数的时候,会检查AVX2指令集是否可用:
\begin{codeline}
  #ifdef __AVX2__ // 仅在支持 AVX2 时编译此部分
    #include <immintrin.h> 
  #else
    #warning "AVX2 not enabled"
  #endif
\end{codeline}
在对应函数,会判断\_\_AVX2\_\_是否被定义,并选择性的运行代码:
\begin{codeline}
  #ifdef __AVX2__
    // ... SIMD 处理代码 ...
  #else
    #warning "AVX2 not supported or enabled. Fail to run simd version"
  #endif 
\end{codeline}


除此之外,基本的函数都提供了运行时异常处理机制,
例如在读取图像文件时,如果文件不存在或格式不支持,函数传入对象不符合要求,则都会抛出异常。

\section{需求和设计思路}
与Project3有显著不同的是,Project3面向对象是纯粹的24位BMP图像,但这个Project的目标是构建一个更通用的图像库。
图像可能有多种通道(Channel,我觉得这个词翻译的不好):RGB三通道的图像是最经典的,而对于灰度图像来说只有一个通道,还有四通道带透明度信息的图像。


图像还分不同的像素类型,RGB就是用三个unsigned char来表示构成一个像素的三种颜色分量。
但除此之外,还有HSV、CMYK、YCbCr等不同的图像类型。以HSV为例,它的三个通道分别是色相、饱和度和明度。
而色相的取值范围是0-360的整数(嗯..?),饱和度和明度的取值范围是0-1的小数,显然不能用char来表示。


在设计核心图像类时,我们面临几个关键问题:如何表示不同类型的图像(灰度图、彩色图、不同颜色空间、不同数据深度)?如何高效且安全地管理像素数据内存?如何提供方便的像素访问和操作接口?
在这个领域,C++的Template非常好用,可以只用写一次函数来支持不同的图像类型。


但实现不同图像类IO是相对麻烦的,
鉴于这次Project主要关注于类的应用,我会用模版类将不同图片形式描述出来,但是IO方法只移植24位的BMP图像。(以及8位灰度图像,我会把它按照24位的逻辑存储)

\subsection{像素表示:Pixel 结构体模板}
在上一个Project中,我们注意到我们可以有两种存储像素的方式。一种是一条线性像素,整齐的摆放着RGBRGBRG...(或者BGR)。
另一种多条线性存储,比如说一条负责存储RRRRR,另外的负责存储其他颜色。乍一看两个存储方式都有优缺点,但是总的来说,第一种的优势更大,
因为相对来说,只处理某一个通道的图像操作是比较少的,况且大多数文件格式也是交错方式存储的。在OpenCV中,cv::Mat也是采用交错存储的方式。
这里,我依旧把像素作为一个单独的结构体来存储。


又由于一个像素内包含的通道数可能不同,数据类型也会不同。为了实现通用性,我们设计了一个 \texttt{Pixel} 结构体模板:
\begin{codeline}
template <typename T, size_t Channels> // T 是数据类型,Channels 是通道数。
struct Pixel {            
public:
    // 编译期检查:确保 T 是算术或枚举,Channels > 0
    static_assert(...);

    T data[Channels]; // 存储通道数据

    // 默认构造函数 (初始化为0)
    Pixel() { /* ... */ }
    // 初始化列表构造函数,例如 pixel = {230, 255, 19}
    Pixel(const std::initializer_list<T>& list) { /* ... */ }

    // 通过 operator[] 访问通道 (带边界检查)
    T& operator[](size_t i);
    const T& operator[](size_t i) const;

    // 静态成员和类型别名,方便获取信息
    static constexpr size_t num_channels = Channels;
    using value_type = T;
};
\end{codeline}
这个设计我们可以通过模版存储不同类型的像素,比如说24位RGB。我们就创建Pixel<unsigned char,3>。
对于灰色图或者说卷积核,可以为Pixel<unsigned char/float,1>。

\subsection{图像容器:Image 模版}
由于上面我的像素使用了模版类,Image也要使用模版类。如果把Pixel看成某种数据类型的话
,我们的任务已经基本转换成了一个矩阵容器。一个矩阵类应该有什么元素?长,宽,以及指向数据数组的指针。
但这三个要素还是显得有点不够用。


首先我们需要保证内存的安全性。考虑到一个图像数据可能会被多个对象共享,所以说销毁一个对象的同时,
必须要确定这个内存是否还有其他对象在用,才能决定是否释放。手动去counter++和delete[]来判断是一个非常折磨的事情,
这里我们直接使用了C++11起支持的智能指针来管理内存。


其次,有些图像,我们只想要其中的一部分,比如说我们只想要图像的中间部分。我们可以通过深拷贝来实现这个功能,
但这样子相对低效。我们可以再用一个指针来指向我们想要的区域的开头。同时,这样的话就需要提前保存图像每一行所占用的字节数。


这样,我们就确定了Image类所包含的数据:
\begin{itemize}
  \item \textbf{宽度}:size\_t 图像的宽度。
  \item \textbf{高度}:size\_t 图像的高度。
  \item \textbf{智能指针}:\texttt{std::shared\_ptr$<$T[]$>$} 指向原始图像数据的开头
  \item \textbf{数据开头指针}:T * 指向期望图像块数据的开头。
  \item \textbf{行大小}:size\_t 每行的字节数,用于计算行偏移。
  \item \textbf{图像类型}:一种枚举类,区分这个图像是RGB、灰度图还是其他类型。
\end{itemize}

\subsection{图像类的操作}
确定了\texttt{Image}类的数据成员后,我们需要给他赋予一些操作,
使其真正成为一个易用且功能完备的图像容器。比如说浅拷贝,深拷贝,访问某一个像素的内容,选中ROI区域等等。


首先,创建一个对象除了default创建一个空图像以及传入长、宽、颜色tag的构造外,
一个关键的构造函数就是创建某一个图像的区域图像。ROI继承了原图像的智能指针,行大小以及图像类型,
修改了长宽以及指向期望图像块的指针。可以在不用拷贝的情况下,直接对原图像进行操作,非常高效。


对于浅拷贝,也就是指针指向同一个数据块。我们甚至可以不用写代码,编译器会自动生成一个default的拷贝构造函数,
帮我们把类里面的元素都拷贝一遍。(为了可读性,也可以专门标注出一行来。)
而对于深拷贝,我们则需要在把成员变量拷贝一遍的同时,用memcpy方法把我们的图像数据拷贝一遍,因为这个方法的性能开销比较大,
所以说用=赋值的,都是default浅拷贝,深拷贝则使用clone()方法。


对于像素数据的访问,通过at(x,y)来返回坐标处的像素引用。也可以通过ptr(y)来返回第y行的指针。


此外,还提供了如\texttt{width()}, \texttt{height()}等众多方法来获取图像的各种属性,这样设计的原因当然是不希望外部能够随意的修改Image类的成员变量。
 其中\texttt{getPixelTypeChannels()}是一个静态\texttt{constexpr}方法,它能根据模板参数\texttt{PixelType}在编译期推断出像素的通道数。


最后销毁一个对象则是用\texttt{release()}方法,这里我们显式的使用了shared\_ptr的reset()方法来告知智能指针这里的引用次数减一,这样我们图像数据的释放就由智能指针来管理了。


集合上述设计要素,这是最终的Image类的设计:
\begin{codeline}
      template <typename PixelType>
      class Image {
      private:
          size_t width_ = 0;
          size_t height_ = 0;
          size_t steps_in_bytes_ = 0; 
          std::shared_ptr<PixelType[]> data_holder_; 
          PixelType* p_data_ = nullptr;             // 指向当前图像/ROI的起始数据
          ColorSpace color_space_ = ColorSpace::UNKNOWN;
          // 内部辅助函数,用于分配内存和初始化元数据
          void allocateAndInit(size_t width, size_t height, ColorSpace cs, 
                               bool initialize_to_default = false);
      public:
          Image() ... * 各种构造函数 *
          // 析构函数 (默认实现,shared_ptr会自动处理内存)
          ~Image() = default;
          // 赋值操作符 (默认浅拷贝)
          Image<PixelType>& operator=(const Image<PixelType>& other) = default;
          Image<PixelType>& operator=(Image<PixelType>&& other) noexcept = default;
          // 创建图像副本 (深拷贝)
          Image<PixelType> clone() const;
          // 将当前图像数据深拷贝到目标图像
          void copyTo(Image<PixelType>& dst) const;
          // 释放资源,图像变为空
          void release();
          // 重新创建图像 (会释放原有数据)
          void create(...);
          * 访问像素函数行指针函数*
          * 获取图像属性函数 *
          // ROI操作符
          Image<PixelType> roi(const Rect& roi_rect);
          Image<PixelType> operator()(const Rect& roi_rect);
          // 重载调整操作符
          Image<PixelType>& operator+=(int value_offset); 
      };
  }
\end{codeline}
写到后面的时候,Image类已经在我眼里感觉比较庞大而且复杂了。
但是,查阅OpenCV官方的cv::Mat Class Reference的使用文档后,我只觉得相形见绌。
他的Mat单论构造方法就有二十几种,
而且他的Mat类的设计也相当复杂,提供了相当多的功能以及sample,的确是一个复杂且优秀的图像库。
\subsection{图像处理方法}
基本原理和上个Project相同,请见reference:Project 3。

\section{如何科学优化}
本次Project很多优化方法和Project2,3所使用的方法相同,因此只是简略的介绍。
\subsection{编译优化}
在CMake构建系统中,针对Release模式配置了以下优化标志:
\begin{itemize}
    \item \texttt{-O3}:启用GCC最高级别的常规优化。
    \item \texttt{-march=native}:指示编译器针对当前构建机器的CPU特性进行优化。
    \item \texttt{-mavx2}:显式启用AVX2 (Advanced Vector Extensions 2) 指令集。
\end{itemize}
对于上述编译优化策略,我们在CMakeLists.txt中进行了如下配置,在非Release模式的时候,程序不会优化:
\begin{codeline}
  set(CMAKE_CXX_FLAGS_RELEASE "${CMAKE_CXX_FLAGS_RELEASE} -O3 -march=native -mavx2")
  set(CMAKE_CXX_FLAGS_DEBUG "${CMAKE_CXX_FLAGS_DEBUG} -g -O0")
\end{codeline}

\subsection{SIMD优化}
对于addBrightness功能,我们提供了两种实现方式:一种就是正常使用的for循环(addBrightness),
另一种则是使用AVX2指令集来实现(addBrightness\_simd)。在实际测试中,AVX2确实带来了部分的性能提升。
\begin{codeline}
  // 使用AVX2指令集实现示例,已简化逻辑的实现细节
  void addBrightness_simd(int value_offset) {
      __m256i offset = _mm256_set1_epi8(value_offset);
      for (size_t i = 0; i < width_ * height_; i += 32) {
          __m256i pixels = _mm256_loadu_si256((__m256i*)&data_[i]);
          pixels = _mm256_add_epi8(pixels, offset);
          _mm256_storeu_si256((__m256i*)&data_[i], pixels);
      }
  }
\end{codeline}


\subsection{OpenMP并行化}
OpenMP依旧伟大,能够几行代码加速几倍。库几乎在所有的处理函数中都使用了OpenMP来实现并行化。
\begin{codeline}
#pragma omp parallel for schedule(static)
for (size_t y = 0; y < height; ++y) {
    // ... 循环处理某些元素 ...
}
\end{codeline}

\subsection{软拷贝以及高效ROI实现}
如前所述,如果不是明确要求深拷贝,Image类的赋值操作符会使用默认的浅拷贝。
这样做的好处是避免了不必要的内存拷贝,同时如果确实需要副本,也可以调用clone()函数来深拷贝。


同时,ROI的实现用了一个非常巧妙的方式。我们指定一个区域,只需要修改图像的宽高和指向数据开头的指针即可,运行成本几乎可以忽略。
具体原理Monad的Report中讲得比较详细,还有配图,可以见reference \cite{Monad2022}。

\subsection{性能测试}
我们的运行条件以及在上面说明,这里,我们延续使用Project3中的分辨率为9725 * 4862的24位无压缩BMP图像进行测试。
我们分成以下三种cases: 完全无任何优化(编译时使用 cmake .. -DCMAKE\_BUILD\_TYPE=debug),
O3 + OpenMP(调用addBrightness函数), O3 + OpenMP + AVX2(调用addBrightness\_simd函数)。程序运行五次时间取平均值。
\begin{figure}[H]
  \centering
  \includegraphics[width=0.8\textwidth]{graph.png}
  \caption{性能测试结果}
  \label{fig:input}
\end{figure}
可以看到,手动实现的SIMD指令集确实能带来性能提升,但是比较小。但无论如何,我们还是看到对比起优化前的速度,程序运行速度提升了大约五倍。


\section{AI辅助了哪些工作}
Move fast and break things.


C++语言实在是太复杂了,很多实用的语法其实我根本没接触过。但是在Gemini的辅助下,我能够极快速的上手C++,虽然有些语法我并不清楚它是怎么实现的,但anyway, it works。这在很多情况下就够用了。
\begin{figure}[H]
  \centering
  \includegraphics[width=0.5\textwidth]{token.png}
  \caption{30万个tokens的上下文!}
  \label{fig:input}
\end{figure}


我这次写Project只使用了一个对话,看看Gemini 2.5Pro面对巨量文本,数据,要求的时候,
它还能不能正确的反应。特别是写类相关的时候,AI能不能基于前面设计好的内容,
快速且正确的移植新方法:事实证明它做到了,而且目测免费的大模型只有它能做到,因为它支持1M的上下文。


比如说,在我设计好大致的Pixel和Image类,以及一些简单的方法后,对话上下文已经来到了十万多。
我叫Gemini帮我移植fliter2D,即通用卷积核的函数,它在第一遍的时候就提供了能够正确运行的函数。
这其实是非常恐怖的,说明它不仅能够正确调用我设计的类的方法,还能够提供一整个读取,卷积,保存的工作链。\sout{真的要失业了吗}。



\section{结语}
在这个Project中,我感受到了C++极其复杂的语言规则,里面有一堆很新很快的函数和方法。
也体会到了写template的方便,对于不同的图像类型,我只用写一次函数,就能支持不同的图像类型,
剩下的全部交给编译器来处理。


同时,shared\_ptr是一个非常伟大的发明。它真的一定程度上解决了跨对象内存管理的难点,
在对象生命周期结束或不再被引用时自动释放所管理的内存的时候,它能够帮我自动释放内存,而不是我手动计数并且释放。
\sout{说不定它早出几十年,Java就会黯然失色。}


总之,我个人对我的库的设计还是比较满意的。
一个有趣的设计点是复用了 \texttt{Image} 类来存储卷积核的数据。这种做法简化了卷积操作的接口,也允许用户方便地创建和检视自定义的卷积核。
当然,这也解释了为何像OpenCV这样的库将其核心图像容器命名为 \texttt{cv::Mat}而非 \texttt{cv::Img},因为矩阵更具通用性,可以自然地包含卷积核这类非典型“图像”的数据结构,从而减少了命名上的潜在歧义。


本项目仍存在一些待完善之处,例如未能支持更多的图像文件格式的IO操作,以及不同颜色空间之间的转换功能。若能补齐这些方面,还能实现更多很炫酷的操作。


同时,与先前Project一样,在DDL结束后我也会将本次Project中涉及到的所有源码开源,放在 https://github.com/BrightonXX/SUSTech-CPP-Project 下。


\textbf{感谢各位能读到这里。}
\newpage % 强制换页
\section*{References}
\begin{thebibliography}{9}

\bibitem{Monad2022} 
Yan W.Q. \textit{CS205 Project5: Matrix Class}. 
Southern University of Science and Technology, 2022. 
[Source code]. Available: 
\href{https://github.com/YanWQ-monad/SUSTech_CS205_Projects}{github.com/YanWQ-monad/CS205\_Project5} 

\bibitem{OpenCVRef}
OpenCV (Open Source Computer Vision Library). \textit{cv::Mat Class Reference}.
Available: \href{https://docs.opencv.org/3.4/d3/d63/classcv_1_1Mat.htmll}{docs.opencv.org/master/class\_Mat.html}

\bibitem{proj3} Brighton 
\textit{Project3: A Simple Image Processor}. 
Southern University of Science and Technology, 2025. 
[Source code]. Available: 
\href{https://github.com/BrightonXX/SUSTech-CPP-Project}{github.com/BrightonXX/SUSTech-CPP-Project} 

\end{thebibliography}
\end{document}