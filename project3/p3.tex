\documentclass[11pt]{article}
\usepackage{amsmath}
\usepackage{listings}
\usepackage{xcolor}
\usepackage{booktabs}
\usepackage{setspace}
\usepackage[hidelinks]{hyperref}
\usepackage{graphicx}
\usepackage{float}
\usepackage[most]{tcolorbox} % 加载 tcolorbox 及其 listings 库
\definecolor{bgcolor}{RGB}{230,230,230}
\definecolor{bgcolor2}{RGB}{240,230,240}
\definecolor{licolor}{RGB}{100,100,100}
\linespread{1.2} 
\usepackage{algorithmic}
\usepackage[a4paper, left=2.5cm, right=2.5cm, top=3cm, bottom=2.5cm]{geometry}
\usepackage{ctex} 
\usepackage{fancyhdr}
\pagestyle{fancy}
\lhead{\textbf{Brighton}}
\chead{BMP Image Processing}
\rhead{CS219 Project3}

\newtcblisting{commandline}{
  breakable,                     % 允许代码块跨页
  colback=bgcolor,               % 背景色为灰色
  colframe=licolor,              % 边框色同背景色(若想显示
  listing only,
  arc=5mm,                       % 设置圆角半径
  boxrule=1pt,                   % 不显示边框线,如需要可调大此数值
  left=6pt, right=6pt, top=6pt, bottom=6pt, % 设置内边距
}
\newtcblisting{codeline}{
  breakable,                     
  colback=bgcolor2,               
  colframe=licolor,              
  listing only,                  
  arc=5mm,                       
  boxrule=1pt,                   
  left=5pt, right=5pt, top=5pt, bottom=5pt, 
  listing options={
    basicstyle=\ttfamily\fontsize{10pt}{11pt}\selectfont,         % 代码字体
    language=bash,                
    showstringspaces=false,       
    aboveskip=3mm, belowskip=3mm, % 上下间距
    columns=flexible,             % 让字体对齐
    keepspaces=true,
    escapeinside={(*@}{@*)},      % 允许使用 LaTeX 命令
    numbers=none,
    frame=none,
  },
  before=\setstretch{0.8},        % 设置行间距 1.5 倍
}

\title{\textbf{A BMP Image Processor}}
\author{} % 清空默认作者区
\date{}   % 清空默认日期区

% 自定义标题布局
\makeatletter
\renewcommand{\maketitle}{
  \begin{center}
    {\LARGE \@title \par} % 标题
    \vspace{1em}
    \textbf{\@author}     % 原作者区(可保留)
    \vspace{1em}
    \begin{tabular}{ll}
      Student ID: & 12312710 \\
      Name:       & 胥熠/Brighton \\
      Course:     & CS219: Advanced Programming \\
      Date:       & \today
    \end{tabular}
    \vspace{0.5em}
    \hrule height 0.8pt % 分隔线
  \end{center}
}
\makeatother

\begin{document}
\thispagestyle{empty}
\maketitle

{
\setstretch{1.2}
\tableofcontents
}
\vspace{3em}
\section{前言}
你说的对,但是「BMP」是由微软自主研发的一种全新无压缩位图存储标准。图像被存储在一个被称作「DIB设备无关位图」的二进制体系中...看到这个Project的第一眼——唉,今年Project好像更新换代了,和往年不太一样了。再仔细看一眼,其实还是换汤不换药——\textbf{\textit{矩阵!}}最后定睛一看,今年的其实要更加友善,毕竟不涉及矩阵的乘法,但是我们依旧可以逐步发掘「优化」的真相。同时在图像处理方面,我还引入了经典Sobel算子(一种简单的卷积操作)来检测图像的边缘。


\section{功能展示}
\subsection{运行环境}
本程序是由C语言编写。运行设备为Lenovo Legion Y9000P运行x86下的Windows 11, 使用Intel i9-13900HX(8 P-cores + 16 E-cores),24GB DDR5运存。使用Powershell通过GCC编译器运行下列命令编译和运行(bmpedit.c放置在path文件夹内):
\begin{commandline}
PS *path*> gcc -o bmpedit bmpedit.c -O3 -march=native -fopenmp -lm
PS *path*> ./bmpedit *Tokens*
\end{commandline}
其中,“-O3”和“-march=native”是可选项,是为了在编译层面加快程序运行速度。“-fopenmp”选项是必要的,因为这个程序中使用了OpenMP并行处理指令。“-lm”是链接的数学库,在绝大多数情况不需要但是也可以加上。

\subsection{程序使用教程}
直接输入./bmpedit或者错误使用指令时,程序会展示使用方法,内容如下:
\begin{commandline}
Usage: ./bmpedit -i input.bmp [-i input2.bmp] -o output.bmp -op operation [args]

Operations:
  add VALUE        - Adjust brightness by adding VALUE to all pixels
  average          - Blend two images by averaging their pixel values
  grayscale        - Convert image to grayscale
  flip h|v         - Flip image horizontally (h) or vertically (v)
  blur [RADIUS]    - Apply blur effect with optional radius (default: 1)
  sobel [THRESHOLD] - Apply Sobel edge detection (default threshold: 100)
  
Examples:
  ./bmpedit -i input.bmp -o output.bmp -op add 50       
  ./bmpedit -i input1.bmp -i input2.bmp -o output.bmp -op average
  ./bmpedit -i input.bmp -o output.bmp -op grayscale    
  ./bmpedit -i input.bmp -o output.bmp -op flip h       
  ./bmpedit -i input.bmp -o output.bmp -op blur 3 
  ./bmpedit -i input.bmp -o output.bmp -op sobel 120
\end{commandline}
\subsection{为图片增添亮度}
在命令行中调用:
\begin{commandline}
./bmpedit -i input.bmp -o output.bmp -op add 200
\end{commandline}
其中文件都在当前目录,input.bmp为输入文件名,output.bmp为输出文件名,add后面数字范围为[1,255]。


输入案例(input.bmp)
\begin{figure}[H]
  \centering
  % 第一个图片所在的区域,宽度设置为文本宽度的45%
  \begin{minipage}[b]{0.45\textwidth}
    \centering
    \includegraphics[width=0.9\textwidth]{demo2.bmp}
   \caption{处理前}
    \label{fig:image1}
  \end{minipage}
  \hfill % 用于在两图片之间插入间距
  % 第二个图片所在的区域,宽度同样为45%
  \begin{minipage}[b]{0.45\textwidth}
    \centering
    \includegraphics[width=0.9\textwidth]{outputdemo1.bmp}
  \caption{处理后}
    \label{fig:image2}
  \end{minipage}
\end{figure}
可以看到在全局加了200点亮度后,一张很黑的图片被处理成了比较亮的图片,同时注意到图片中间发光区域的细节消失了。



\subsection{平均两张图片}
在命令行中调用:
\begin{commandline}
./bmpedit -i input.bmp -i input2.bmp -o output.bmp -op average
\end{commandline}
对两张相同长宽的照片进行像素级别的平均,其对应像素RGB是两个图片对应像素RGB值的算数平均数。


输入案例(input.bmp,input2.bmp)
\begin{figure}[H]
  \centering
  % 第一个图片所在的区域,宽度设置为文本宽度的45%
  \begin{minipage}[b]{0.45\textwidth}
    \centering
    \includegraphics[width=0.6\textwidth]{demoqq.bmp}
   \caption{一只可爱的企鹅}
    \label{fig:image1}
  \end{minipage}
  \hfill % 用于在两图片之间插入间距
  % 第二个图片所在的区域,宽度同样为45%
  \begin{minipage}[b]{0.45\textwidth}
    \centering
    \includegraphics[width=0.6\textwidth]{demo1.bmp}
  \caption{一条可爱的鲨鱼}
    \label{fig:image2}
  \end{minipage}
\end{figure}

\begin{figure}[H]
  \centering
  \includegraphics[width=0.4\textwidth]{outputav.bmp}
  \caption{平均结果}
  \label{fig:example}
\end{figure}
可以看到两张图片平均后原本的颜色都变淡了,且重合部分可以同时看到两个图片的细节,产生了半透明叠加的效果。


\subsection{让图片变成灰色调}
在命令行中调用:
\begin{commandline}
./bmpedit -i input.bmp -o output.bmp -op grayscale 
\end{commandline}
将彩色图像转换为灰度图像,模拟黑白照片效果。本程序采用常用的加权平均法(Luminance method: $Gray = 0.299 \times R + 0.587 \times G + 0.114 \times B$)。


输入案例(input.bmp)
\begin{figure}[H]
  \centering
  % 第一个图片所在的区域,宽度设置为文本宽度的45%
  \begin{minipage}[b]{0.45\textwidth}
    \centering
    \includegraphics[width=0.8\textwidth]{demo6.bmp}
   \caption{一张五彩斑斓的图像}
    \label{fig:image1}
  \end{minipage}
  \hfill % 用于在两图片之间插入间距
  % 第二个图片所在的区域,宽度同样为45%
  \begin{minipage}[b]{0.45\textwidth}
    \centering
    \includegraphics[width=0.8\textwidth]{outputgr.bmp}
  \caption{转换后的灰度图像}
    \label{fig:image2}
  \end{minipage}
\end{figure}


\subsection{水平竖直翻转}
在命令行中调用:
\begin{commandline}
./bmpedit -i input.bmp -o output1.bmp -op flip h
./bmpedit -i input.bmp -o output2.bmp -op flip v
\end{commandline}
将图像左右镜像翻转,或者上下镜像翻转。


输入案例(input.bmp)
\begin{figure}[H]
  \centering
  \includegraphics[width=0.5\textwidth]{demo7.bmp}
  \caption{原始图像}
  \label{fig:example}
\end{figure}

输出案例(output1.bmp、output2.bmp)
\begin{figure}[H]
  \centering
  % 第一个图片所在的区域,宽度设置为文本宽度的45%
  \begin{minipage}[b]{0.45\textwidth}
    \centering
    \includegraphics[width=0.9\textwidth]{demo8.bmp}
   \caption{左右翻转后的图像}
    \label{fig:image1}
  \end{minipage}
  \hfill % 用于在两图片之间插入间距
  % 第二个图片所在的区域,宽度同样为45%
  \begin{minipage}[b]{0.45\textwidth}
    \centering
    \includegraphics[width=0.9\textwidth]{demo9.bmp}
  \caption{上下翻转后的图像}
    \label{fig:image2}
  \end{minipage}
\end{figure}


\subsection{为图片添加模糊效果}
在命令行中调用:
\begin{commandline}
./bmpedit -i input.bmp -o output.bmp -op blur 4
\end{commandline}
对图像应用简单的盒子模糊(Box Blur)效果,使图像看起来模糊或失焦。其中“4”代表模糊半径为4格像素(即处理 $(2*4+1) \times (2*4+1) = 9 \times 9$ 的邻域)。可以换成其他任意大小的正整数(程序内限制了上限为10)。

\begin{figure}[H]
  \centering
  \includegraphics[width=0.8\textwidth]{demo10.bmp}
  \caption{原始图像}
  \label{fig:example}
\end{figure}

\begin{figure}[H]
  \centering
  \includegraphics[width=0.8\textwidth]{outputbl.bmp}
  \caption{处理后图像}
  \label{fig:example}
\end{figure}

\subsection{基于Soble算子的边缘检测}
在命令行中调用:
\begin{commandline}
./bmpedit -i input.bmp -o output.bmp -op sobel 200
\end{commandline}
对图像应用简单的Soble算子,检测出图像中RGB信息变化剧烈的边缘部分。其中“200”是设定的阈值,高于阈值的部分会被认为是边缘并输出黑色,否则输出白色。

\begin{figure}[H]
  \centering
  % 第一个图片所在的区域,宽度设置为文本宽度的45%
  \begin{minipage}[b]{0.45\textwidth}
    \centering
    \includegraphics[width=0.8\textwidth]{demo20.bmp}
   \caption{原始图像}
    \label{fig:image1}
  \end{minipage}
  \hfill % 用于在两图片之间插入间距
  % 第二个图片所在的区域,宽度同样为45%
  \begin{minipage}[b]{0.45\textwidth}
    \centering
    \includegraphics[width=0.8\textwidth]{outputso.bmp}
  \caption{处理后的图像}
    \label{fig:image2}
  \end{minipage}
\end{figure}
可以看到处理后的图像保留了原图像的主要轮廓和纹理边缘。

\subsection{错误输入识别}
程序能够识别一些常见的用户输入错误或文件问题,并给出相应的提示信息,避免程序异常退出或产生错误结果,具有良好的Robustness。
\begin{commandline}
# 缺少操作
./bmpedit -i demo5.bmp -o out.bmp            
Error: No operation specified
*Display Usage*
  
# 缺少必要参数 (如输出文件)
PS *path*> ./bmpedit -i input.bmp -op add 50
Error: No output file specified
Usage: ./bmpedit ...

# 操作参数不足 (如 'add' 缺少数值)
PS *path*> ./bmpedit -i input.bmp -o out.bmp -op add
Error: 'add' operation requires a value

# 无效操作名称
PS *path*> ./bmpedit -i input.bmp -o out.bmp -op rotate 90
Error: Unknown operation 'rotate'

# 文件无法打开或不存在
PS *path*> ./bmpedit -i non_existent.bmp -o out.bmp -op grayscale
Error: Cannot open file non_existent.bmp
Error: Could not load input file non_existent.bmp

# 文件格式不支持 (非"BM"签名, 或非24位, 或压缩格式)
PS *path*> ./bmpedit -i not_a_bmp.jpg -o out.bmp -op grayscale
Error: Not a BMP file (invalid signature)
PS *path*> ./bmpedit -i indexed_color.bmp -o out.bmp -op grayscale
Error: Only 24-bit uncompressed BMP files are supported

# 'average' 操作缺少第二个输入文件
PS *path*> ./bmpedit -i input1.bmp -o out.bmp -op average
Error: 'average' operation requires two input images

# 'average' 操作两输入文件尺寸不匹配
PS *path*> ./bmpedit -i input1.bmp -i different_size.bmp -o out.bmp -op average
Error: Images must have the same dimensions for averaging
Error: Operation failed

# 程序成功运行时提示 (blur 和 sobel 操作还会返回运行时间供参考)
PS *path*> ./bmpedit -i demo12.bmp -o outputtemp.bmp -op blur 5   
Blur operation took 2.588488 seconds
Operation 'blur' completed successfully. Output saved to outputtemp.bmp
\end{commandline}
这些检查能够保证程序能够被正确使用,并且给出对应错误提示方便用户纠正。

\section{功能实现}
\subsection{24位无压缩BMP文件的存储结构}
BMP(Bitmap)是一种简单的图像格式,而其中24位无压缩BMP则是其中的相对好处理的一种格式。在这个Project中,我将只关注24位无压缩BMP文件。\textbf{因此在下述的报告中,所有BMP都指24位无压缩BMP文件。}


一个标准的24位无压缩BMP文件主要由三部分构成:14字节的文件头、通常为40字节的信息头和像素数据。
% BMP文件头表格
\begin{table}[htbp]
  \centering
  \caption{BMP文件头 (BMP Header - 14字节)}
  \begin{tabular}{|c|c|l|}
    \hline
    \textbf{偏移量} & \textbf{大小(字节)} & \textbf{描述} \\
    \hline
    0x00 & 2 & 文件类型标识符(必须为"BM",0x42 0x4D) \\
    \hline
    0x02 & 4 & 文件大小(以字节为单位) \\
    \hline
    0x06 & 2 & 保留,通常设为0 \\
    \hline
    0x08 & 2 & 保留,通常设为0 \\
    \hline
    0x0A & 4 & 从文件头到像素数据的偏移量 \\
    \hline
  \end{tabular}
  \label{tab:bmp_header}
\end{table}
这部分位于文件的开头,提供了文件的基本标识和信息。
% 信息头表格
\begin{table}[htbp]
  \centering
  \caption{信息头 (DIB Header - 40字节)}
  \begin{tabular}{|c|c|l|}
    \hline
    \textbf{偏移量} & \textbf{大小(字节)} & \textbf{描述} \\
    \hline
    0x0E & 4 & 信息头大小(对于BITMAPINFOHEADER为40) \\
    \hline
    0x12 & 4 & 图像宽度(像素) \\
    \hline
    0x16 & 4 & 图像高度(像素,可为负值) \\
    \hline
    0x1A & 2 & 色彩平面数(必须为1) \\
    \hline
    0x1C & 2 & 每像素位数(对于24位BMP为24) \\
    \hline
    0x1E & 4 & 压缩方式(0表示无压缩) \\
    \hline
    0x22 & 4 & 图像大小(可设为0表示使用默认值) \\
    \hline
    0x26 & 4 & 水平分辨率(像素/米) \\
    \hline
    0x2A & 4 & 垂直分辨率(像素/米) \\
    \hline
    0x2E & 4 & 调色板颜色数(0表示使用最大值) \\
    \hline
    0x32 & 4 & 重要颜色数(通常为0) \\
    \hline
  \end{tabular}
  \label{tab:dib_header}
\end{table}
紧随文件头之后,描述了图像的具体属性,如尺寸、颜色深度等。


对于像素部分,24位无压缩BMP就是指图片中一个像素占24位,也就是3字节。其中三个字节分别存储蓝色、绿色、红色,也就是经典的RGB存储\sout{(额,不应该叫BGR吗)}。一个unsigned字节能够表示的数字范围为0-255,数字越小代表这个颜色越暗。所以RGB(0,0,0)代表的是黑色,而RGB(255,0,0)代表的是纯红色。


像素按行存储,而且是\textbf{从下到上}存储。每行像素数据按4字节对齐。例如一行中有五个像素,则一行的数据会有$5\times 3=15$个字节,为了字节对齐4的倍数,BMP文件在这一行后会填充一byte的对齐数据,因此在读取的时候要计算好Padding量,在一行结束时谨慎确定下一行开始位置。
\subsection{数据存储方式}
在这里,我们为了方便使用fread读取,我们定义了四个结构体,分别为: BMPFileHeader(14字节文件头)、BMPInfoHeader(40字节信息头)、Pixel(24位像素)和BMPImage(图片总封装)。


其中,BMPFileHeader和BMPInfoHeader的封装如下,其顺序符合上述表格:
\begin{codeline}
 typedef struct {
     unsigned short bfType;       // Magic identifier "BM" (0x4D42)
     unsigned int   bfSize;       // File size in bytes
     unsigned short bfReserved1;  // Reserved
     unsigned short bfReserved2;  // Reserved
     unsigned int   bfOffBits;    // Offset to image data
 } BMPFileHeader;
 
 typedef struct {
     unsigned int   biSize;          // Header size
     int            biWidth;         // Width of the image
     int            biHeight;        // Height of the image
     unsigned short biPlanes;        // Color planes
     unsigned short biBitCount;      // Bits per pixel
     unsigned int   biCompression;   // Compression type
     unsigned int   biSizeImage;     // Image size in bytes
     int            biXPelsPerMeter; // X resolution
     int            biYPelsPerMeter; // Y resolution
     unsigned int   biClrUsed;       // Colors used
     unsigned int   biClrImportant;  // Important colors
 } BMPInfoHeader;
\end{codeline}

对于一个像素,包含红绿蓝三种颜色,因此也只需要封装三个unsigned char。
\begin{codeline}
typedef struct {
     unsigned char blue;
     unsigned char green;
     unsigned char red;
 } Pixel;
\end{codeline}

而对于整体BMPImage封装,则需要稍微小心。这里不采用二维数组(双指针)存储像素,而是采用线性存储,即数组的第[0, width-1]个元素代表第一行,[width, 2*width-1]代表第二行...访问 (x, y) 处的像素通过计算索引 $y \times width + x$ 实现。也可以定义一个宏#define PIXEL\_AT(image, x, y) ((image)->pixels + (y) * (image)->width + (x)) 来方便计算索引。


其实对于矩阵,采用线性存储是普遍的常识,也是后续SIMD实现的必要数据结构。如果使用双重指针,不仅多很多次不必要的解指针操作,内存不连续还会导致CPU根本没有办法进行并行操作。同时显而易见的,线性存储只用一次malloc,这为内存管理提供了很大的方便。


以及,存储矩阵的长宽属性,我们应该使用$size\_t$,在比int范围大一倍的同时免去了检查参数负数的必要。集齐以上要素,我的BMP封装的设计为:
\begin{codeline}
typedef struct {
     BMPFileHeader fileHeader;
     BMPInfoHeader infoHeader;
     Pixel* pixels; 
     size_t width;
     size_t height;
} BMPImage;
\end{codeline}
符合直觉且合理高效的存储了一个BMP图像。

\subsection{内存对齐}
在读取文件的时候,我们必须要保证文件的每一个字节都被映射到适当的位置。但是C/C++的结构体成员内存对齐是一个相对混沌邪恶的东西,它可能会在某些2字节的变量后面插入两个填充字节以保证4字节对齐(例如在 $BMPFileHeader$ 的 $bfType$ 和 $bfSize$ 之间)。这样会导致 $sizeof(BMPFileHeader)$ 不等于14字节,$fread$ 读取的数据就会错位。


因此对于两种文件头封装结构,我们必须使用禁用自动对齐,保证结构体成员紧密排列,与文件格式完全一致。
\begin{codeline}
#pragma pack(push, 1) // 禁用自动对齐
 typedef struct { /* Header members */ } BMPFileHeader;
 typedef struct { /* Info header members */ } BMPInfoHeader;
#pragma pack(pop)   // 恢复自动对齐
\end{codeline}
保证了正确对齐后,我们就可以直接使用 fread 和 fwrite 直接读写结构体。

\subsection{图片处理}
现在,我们成功的将一张图片的数据导入到了结构体中,终于到了这个程序最核心,但也是相对简单的一部分,处理图片。

\paragraph{亮度调整 (addBrightness):}
为了调整亮度,我们只需要for循环,将每一个Pixel元素的红绿蓝三个值加上某一个数值,大于255的数字设置成255就可以了。上面事例图的高光部分消失就是因为加上一个数字时候,那一片区域的RGB都超出阈值都被设置成上限255,所以细节丢失了。

\paragraph{图像平均 (averageImages):}
首先检查两张输入图像的尺寸(宽度和高度)是否完全一致,如果不一致则报错退出。如果尺寸相同,就创建一个新的空白图像作为结果。然后遍历所有像素位置计算RGB算数平均值即可。


\paragraph{灰度化 (convertToGrayscale):}
利用加权平均数 0.299R + 0.587G + 0.114B 计算亮度(Luminance Preserving,原理基于人眼对光的敏感度),然后将这个数值在同时赋给RGB三色,呈现出来的就是对应亮度的灰色调。


\paragraph{图像翻转 (flipImage):}
原理非常简单,根据是H还是V进行行或列内的数列翻转操作。


\paragraph{盒子模糊 (blurImage):}
$blur$操作的原理是对一个像素而言,后模糊信息来自于距离为半径窗口范围内的像素的平均值。也就是说假如说假如说我们的模糊半径为2,则我们计算以此像素为中心,边长为5的正方形像素的RGB平均值。这么看我们的算法时间复杂度为$O(r^2*pixels)$,我们可以通过滑动窗口算法来优化。即计算完第一个像素后,右侧的像素只需要在左侧计算结果的基础上,减去最左侧一边,加上最右边的数据即可,优化后,算法的时间复杂度为$O(r*pixels)$。还可以进一步优化吗,可以的!

\paragraph{Sobel 边缘检测 (detectEdges):}
Sobel 算子是一种经典的基于一阶导数的边缘检测方法。原理是通过卷积核(Sobel算子)计算水平和竖直方向上的梯度,再通过几何平均值(在这里为了计算效率我使用了绝对值的算数平均)确定是否大于某个阈值,如果大于则赋为黑色RGB(0,0,0),如果小于则赋为白色RGB(255,255,255),于是我们就可以获得白底黑边的边缘图。下面是Sobel算子的矩阵表示:

 $$ G_x = \begin{bmatrix} -1 & 0 & +1 \\ -2 & 0 & +2 \\ -1 & 0 & +1 \end{bmatrix} \quad G_y = \begin{bmatrix} -1 & -2 & -1 \\ 0 & 0 & 0 \\ +1 & +2 & +1 \end{bmatrix} $$

\section{如何科学优化}

\subsection{线性存储像素}
线性存储像素的优势已经在上面提及了,它不仅避免了多次malloc和解指针,还是后续并行操作的基石,还可以改善Cache命中率,因此再次强调一下。
\subsection{编译器优化}
首先,所有不开O3优化的性能分析都没有任何意义。其次,C语言基础语法性能不够,打算手写SIMD?我们在这个Project中涉及的算法都是矩阵加法,由于我们线性存储像素,所以一切操作在编译器面前都是$Vector_A = Vector_B + Vector_C$的形式,和向量点乘一样属于编译器看得懂,能做优化的类型。因此在gcc的-march=native的优化下,能够生成SIMD技术加持的机器码,因此也没必要手写了。这是addBrightness方法中的部分逆向机器码:

\begin{commandline}
    addBrightness:这是外部调用的主函数接口。任务是设置环境、检查参数,并启动OpenMP。
    4035e0:	callq  4039c0 <GOMP_parallel>  ; 调用OpenMP并行区域
    addBrightness._omp_fn.0: 这是由编译器为 OpenMP 并行区域生成的工作函数。
    4015f5:	vpmovzxbw %xmm7,%ymm10       ; 零扩展8位到16位
    401635:	vpaddd %ymm14,%ymm2,%ymm2    ; 向量加法
    40163a: vinserti128...               ; 合并成32字节YMM5
    40164f: vmovdqu -0x40(%r8),%xmm6     ; 预取下一块数据
    40165b: vpmovzxbw %xmm7,%ymm10       ; 8位→16位无符号扩展
    40166e: vpaddd %ymm14,%ymm2,%ymm2    ; 32位加法
    4016a0:	vpminsd %ymm13,%ymm9,%ymm9   ; 向量最小值饱和
    4017ac:	vpackuswb %ymm5,%ymm2,%ymm5  ; 打包32位到8位
    ...
\end{commandline}
在gcc生成的SIMD中,一如既往的将标量计算转换成向量计算,一次迭代处理32字节。可以看到在这种相对简单直白的任务中,gcc能够应付过来。


但是需要注意的是当任务变得复杂,比如很多分支或者内存不规则的情况,gcc很可能就不能理解你在干什么进而无法生成对应SIMD了,因此优化也不能全部指望编译器,\sout{或者说要写编译器能看懂的代码}。
\subsection{并行化操作}
OpenMP伟大,我们只需要在需要并行化的循环前加上一行 \#pragma omp parallel for (args),然后在编译的时候加上 -fopenmp 连接上OpenMP的库就可以享受并行给程序带来的速度。使用OpenMP需要注意的点就是尽量减少数据竞争,就比如说单提向量点乘加到一个变量的操作,可以优化成点乘分散加到一个向量上最后相加。多个线程竞争着写入一个变量会导致一定的性能削减。在我们的图像处理中,每个像素都是相对独立的,因此这种顾虑比较小。
\subsection{避免重复索引计算}
在blurImage的滑动窗口计算中有两个双重循环,大概长这样:
\begin{codeline}
    // 水平方向模糊 
     for (int i = 0; i < height; i++) {
        for (int j = 0; j < width; j++) {
            * 计算水平窗口的像素和 *
            long long sumRed = 0, sumGreen = 0, sumBlue = 0, int count = 0; ...(省略中间计算代码)
            * 存储结果到临时缓冲区 *
            temp[i * width + j].red = (unsigned char)(sumRed / count);
            temp[i * width + j].green = (unsigned char)(sumGreen / count);
            temp[i * width + j].blue = (unsigned char)(sumBlue / count);
                 注意这里的计算冗余
        }
    }
\end{codeline}
我们发现,在存储结果到临时缓冲区的时候,一个forJ循环要计算3*width次$i\times width$,这个计算显然是冗余的,我们可以这样优化:
\begin{codeline}
     for (int i = 0; i < height; i++) {
        int index_temp = i * width;
        for (int j = 0; j < width; j++) {
            * 存储结果到临时缓冲区 *
            temp[index_temp + j].{red,green,blue} = (unsigned char)(sum{Color} / count);
        }
    }
\end{codeline}
这样提前计算好索引开始的位置,我们一次forJ循环就只用计算一次$i\times width$,这个双重循环总共节省了大约3*pixels次乘法计算。


测试性能,发现这种操作好像没有带来时间上的优化,似乎是O3优化帮我们做了,但这种思想还是值得一提。
\subsection{多种封装方式}
我们注意到,其实Pixel封装像素数据对于某些并行操作而言,是一种相对不利的数据结构。比如说,我们想要让某张图像的颜色变得更绿,假如说我们用三条数组分别存储RGB颜色数据。这样的话,对于每一种颜色的内存空间都是连续的了,变绿也就变成真正的向量加法操作了,效率肯定会有提升。


但这也不一定对所有并行操作有利,例如Sobel算子,RGB单独存储反而吃了不连续和多次读写的亏。因此我这里只是抛砖引玉一下,在实际应用中需要按照需求和实际性能表现选择不同的封装方式,\sout{才不是因为没时间了。}具体可以搜索关键词:AoS: Array of Structures vs SoA: Structure of Arrays。


\subsection{性能对比}
为了量化各个不同优化手段的加速效果,我们以计算量最大的 blur (Radius = 5)为例进行计时比较。测试环境同上(i9-13900HX),计时方法采用Project 2中实现的跨平台高精度计时器,仅计算核心计算时间不计入文件IO时间。

\begin{figure}[H]
  \centering
  \includegraphics[width=0.6\textwidth]{demo12.bmp}
  \caption{实验使用的图像,为展示已压缩处理}
  \label{fig:example}
\end{figure}


这里我们设计了四种情况,分别是不使用编译器优化,单采用-O3 -march=native优化(SIMD指令),单使用OpenMP,SIMD+OpenMP四种不同优化类型组合。处理一张大小为$9725\times 4862 \approx 47M$像素的图片(上图),只记录实际处理时间,不记录文件IO时间,重复五次记录平均值。这是实验结果:

\begin{figure}[H]
  \centering
  \includegraphics[width=0.8\textwidth]{sta2.png}
  \caption{47M}
  \label{fig:example}
\end{figure}

感觉加了OpenMP后,SIMD好像没有起到什么优化效果。我们再将原图长宽放大三倍(数据量翻九倍)看看,现在图片大小为$29175\times 14586 \approx 426M$,占用硬盘空间已经达到了1.16G,可以说是一个巨量的数据了:

\begin{figure}[H]
  \centering
  \includegraphics[width=0.8\textwidth]{sta1.png}
  \caption{426M}
  \label{fig:example}
\end{figure}

可以看到,OpenMP + SIMD的组合在计算密集型任务中确实是表现最佳的。而且数据量非常大的时候也能够和单OpenMP拉开一些差距,这可能是因为SIMD的优化效果常数比较大。在426M的数据量下,OpenMP + SIMD比无优化的速度快了11倍左右,还是相当可观的。


我们再用Intel VTune Profiler 观察一下运行瓶颈:

\begin{figure}[H]
  \centering
  \includegraphics[width=0.8\textwidth]{ana.png}
  \caption{分析结果}
  \label{fig:example}
\end{figure}

首先最大的变化,我们发现Total Thread Count变成了35个,可以看到运行中段处理照片的时候,OpenMP帮助我们开了很多线程并行处理图片。我们也发现Memory Bound很小,主要程序慢在Divider(取平均值那一步)和Port Utilization(很多线程同时向Sum写入数据,需要等待)。总的来看优化做的还算合理。
\section{AI辅助了哪些工作}

\subsection{程序主体逻辑}
这个Project比较LLM友好,我的程序主体架构是由Claude 3.7 Sonnet提供的,它主要帮我完成了程序的IO和Tokenize部分。它也帮我写了一些方法和结构体,但写的不好。首先它用的是二维数组存储,再其次它用的是int存储长宽。我在他的基础上进行了封装体的改进,一些新方法的移植,以及OpenMP等进一步的优化。
\subsection{加速撰写报告}
我有一个神奇的观点——在LLM逐渐变强的时代下,我们必须要建立一个自己的语料库。


谷歌最新的Gemini 2.5 Pro已经支持了100M Token的上下文,最长至65536Token的输出。这种量变会造成质变,我可以把前两个Project的latex丢进去,把我的程序源码丢进去,甚至把我写的一些文章丢进去(模仿语言风格)让他来写Project3报告。事实上我虽然没有直接粘贴他的报告,但也边看着AI的输出边自己写,起到了一个参考的作用,而且它还自动帮我补全了部分latex,的确能够加速报告的完成。
\section{结语}
本Project实现了一个命令行24位无压缩BMP图像处理器,可以支持很多功能例如灰色调和边缘处理,同时也探索了如何加速图片处理的过程。

本Project还有很多可以改进的地方,例如在优化领域做的工作还是太浅了,为了效率肯定还是要手写原生的SIMD的。但是由于本Project时间被期中周覆盖,工作量太大了实在是无法完成。但作为一次学习来说还是相对成功的,我探索了数据结构、优化方法和SIMD并行计算等知识,从不同角度思考了如何加速图片处理。


同时,与Project 1和Project 2一样,在DDL结束后我也会将本次Project中涉及到的所有源码开源,放在 https://github.com/BrightonXX/SUSTech-CPP-Project 下。


\textbf{感谢各位能读到这里。}
\newpage % 强制换页
\section*{References}
\begin{thebibliography}{9}

\bibitem{Monad2022} 
Yan W.Q. \textit{CS205 Project4: Matrix Multiplication}. 
Southern University of Science and Technology, 2022. 
[Source code]. Available: 
\href{https://github.com/YanWQ-monad/SUSTech_CS205_Projects}{github.com/YanWQ-monad/CS205\_Project4} 



\end{thebibliography}
\end{document}